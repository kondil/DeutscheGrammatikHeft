\documentclass[a4paper,10pt]{article}
%\documentclass[a4paper,10pt]{scrartcl}

\usepackage{xltxtra}
\usepackage{xgreek}
\usepackage{fontspec}
\usepackage{color}
\usepackage[toc,page]{appendix}
\usepackage[toc]{glossaries}
\makeglossaries
\setromanfont{FreeSerif}
% \setsansfont{FreeSans}
% \setmonofont{FreeMono}
% \setmainfont[Mapping=tex-text]{GFS Didot}

% \setromanfont[Mapping=tex-text]{Linux Libertine O}
% \setsansfont[Mapping=tex-text]{DejaVu Sans}
% \setmonofont[Mapping=tex-text]{DejaVu Sans Mono}


\newcommand{\addVerb}[8]{- #1 \\\begin{tabular}{l l l}  ich & #2 & #3 \\
  du & #2 & #4 \\
  er, sie, es & #2 & #5 \\
  wir & #2 & #6 \\
  ihr & #2 & #7 \\
  Sie & #2 & #8 \\
 \end{tabular} 
 }
 
 \newcommand{\addNormalVerb}[2]{ \addVerb{#1}{#2}{-e}{-st}{-t}{-en}{-t}{-en} }

  \newcommand{\addBVerb}[2]{- #1 \\\begin{tabular}{l l l}  ich & #2 & -e \\
    du & #2 & -\textcolor{red}{e}st \\
    er, sie, es & #2 & -\textcolor{red}{e}t \\
    wir & #2 & -en \\
    ihr & #2 & -\textcolor{red}{e}t \\
    Sie & #2 & -en \\
  \end{tabular} 
  }
  
  \newcommand{\addCVerb}[2]{ \addVerb{#1}{#2}{-e}{-t}{-t}{-en}{-t}{-en} }
  
  \newcommand{\addDVerb}[3]{- #1 \\\begin{tabular}{l l l}  ich & #2 & -e \\
    du & #3 & -st \\
    er, sie, es & #3 & -t \\
    wir & #2 & -en \\
    ihr & #2 & -t \\
    Sie & #2 & -en \\
  \end{tabular} 
  }
 
  
  \newcommand{\addGVerb}[2]{- #1 \\\begin{tabular}{l l l}  ich & #2 & -e \\
    du & #2 & -st \\
    er, sie, es & #2 & -t \\
    wir & #2 & -n \\
    ihr & #2 & -t \\
    Sie & #2 & -n \\
  \end{tabular} 
  }
 
\title{}
\author{}
\date{}

\begin{document}
\maketitle

\begin{appendix}
 
 \section{Verben - Ρήματα}
 
 Στο παρόν παράρτημα παραθέτουμε τις συζηγίες Ρημάτων και ποια ρήματα ανήκουν σε αυτές.
 
 \subsection{Regelmaessige Verben - Ομαλά Ρήματα}
 
 \addNormalVerb{Wohnen}{wohn} \\ 
 \addNormalVerb{Kommen}{komm}
 
 \subsection{Unregelmaessige Verben - Ανώμαλα Ρήματα}
 
 \subsubsection{B Verben}
 Ρήματα τα οποία χρειάζονται ένα -e- ανάμεσα στο θέμα (-t, -d, -m, -n, διπλό σύμφωνο) και την κατάληξη για λόγους ευφωνίας. \\ \newline
 \addBVerb{Arbeiten}{arbeit} \\ \newline
 Rechnen, Atmen
 
 \subsubsection{C Verben}
 Ρήματα των οποίων το θέμα λήγει σε -ss, -s, -z, -tz, -x. \\
 \addCVerb{Heissen}{heiss} \\ \newline
 Essen, Reisen, Tanzen, Sitzen, Boxen
 \subsubsection{D Verben}
 Ρήματα των οποίων το θέμα αλλάζει στο $2^o$ και $3^ο$ πρόσωπο από $e \rightarrow ie$. \\
 \addDVerb{Sehen}{seh}{sieh}
 
 \subsubsection{E Verben}
 Ρήματα των οποίων το θέμα αλλάζει στο $2^o$ και $3^ο$ πρόσωπο από $e \rightarrow i$. \\
 \addDVerb{Sprechen}{sprech}{spr\textcolor{red}{i}ch} \\
 \addDVerb{Nehmen}{nehm}{n\textcolor{red}{i}mm} \\
 
 \subsubsection{F Verben}
 Ρήματα των οποίων το θέμα αλλάζει στο $2^o$ και $3^ο$ πρόσωπο από a $\rightarrow$ ae. \\
 \addDVerb{Fahren}{fahr}{faehr}
 
 \subsubsection{G Verben}
 Ρήματα τα οποία τελειώνουν σε -n (-eln) \\
 \addGVerb{Basteln}{bastel}
 
 
 
 \newcommand{\nge}[2]{\newglossaryentry{#1}{ name={#1}, description={#2} }}
 \nge{Sein}{Είμαι}
 \nge{Haben}{Έχω}
 \nge{Werden}{Γίνομαι}
 \nge{Wissen}{Γνωρίζω}
 \nge{Moegen}{Επιθυμώ, μου αρέσει, θέλω}
 
 
 
 \section{ ... Verben}
 \begin{tabular}{l l l l l l}
  & \gls{Sein} & \gls{Haben} & \gls{Werden} & \gls{Wissen} & \gls{Moegen} \\
  ich & bin & habe & werde & weiss & mag \\
  du & bist & hast & wirst & weisst & magst \\
  er, sie, es & ist & hat & wird & weiss & mag \\
  wir & sind & haben & werden & wissen & moegen \\
  ihr & seid & habt & werdet & wisst & moegt \\
  Sie & sind & haben & werden & wissen & moegen \\
 \end{tabular}

 
 \section{Modalverben}
 
 \begin{tabular}{l l l l l l l}
  & Moechten & Muessen & Koennen & Duerfen & Wollen & Sollen \\
  ich & moechte & muss & kann & darf & will & soll \\
  du & moechtest & musst & kannst & darfst & willst & sollst \\
  er, sie, es & moechte & muss & kann & darf & will & soll \\
  wir & moechten & muessen & koennen & duerfen & wollen & sollen \\
  ihr & moechtet & muesst & koennt & duerft & wollt & sollt \\
  Sie & moechten & muessen & koennen & duerfen & wollen & sollen \\
 \end{tabular}

 
 
\end{appendix}

\printglossaries

\end{document}
