
\begin{appendix}
 
 \section{Verben - Ρήματα}
 
 Στο παρόν παράρτημα παραθέτουμε τις συζηγίες Ρημάτων και ποια ρήματα ανήκουν σε αυτές.
 
 \subsection{Regelmaessige Verben - Ομαλά Ρήματα}
 
 \addNormalVerb{\gls{Wohnen}}{wohn} \\ 
 \addNormalVerb{\gls{Kommen}}{komm} \\
 
 \gls{Backen}, \gls{Bekommen}, \gls{Bestellen}, \gls{Bezahlen}, \gls{Brauchen}, \gls{Buchstabieren}, \gls{Erkennen}, \gls{Erzaehlen}, \gls{Entschuldigen}, \gls{Fragen}, \gls{Funktionieren}, \gls{Gehen}, \gls{Glauben}, \gls{Kennen}, \gls{Kochen}, \gls{Korrigieren}, \gls{Lernen}, \gls{Machen}, \gls{Meinen}, \gls{Reisen}, \gls{Sagen}, \gls{Schmecken}, \gls{Schreiben}, \gls{Spielen}, \gls{Spuelen}, \gls{Stimmen}, \gls{Studieren}, \gls{Trinken}, \gls{Ueben}
 
 \subsection{Unregelmaessige Verben - Ανώμαλα Ρήματα}
 
 \subsubsection{B Verben}
 Ρήματα τα οποία χρειάζονται ένα -e- ανάμεσα στο θέμα (-t, -d, -m, -n, διπλό σύμφωνο) και την κατάληξη για λόγους ευφωνίας. \\ \newline
 \addBVerb{\gls{Arbeiten}}{arbeit} \\ \newline
 \gls{Antworten}, \gls{Atmen}, \gls{Bieten}, \gls{Entscheiden}, \gls{Kosten}, \gls{Rechnen}, \gls{Warten}, 
 
 \subsubsection{C Verben}
 Ρήματα των οποίων το θέμα λήγει σε -ss, -s, -z, -tz, -x. \\
 \addCVerb{\gls{Heissen}}{heiss} \\ \newline
 \gls{Essen}, \gls{Reisen}, \gls{Tanzen}, \gls{Sitzen}, \gls{Boxen}
 \subsubsection{D Verben}
 Ρήματα των οποίων το θέμα αλλάζει στο $2^o$ και $3^ο$ πρόσωπο από $e \rightarrow ie$. \\
 \addDVerb{Sehen}{seh}{sieh} \\
 
 \gls{Lesen}, 
 
 \subsubsection{E Verben}
 Ρήματα των οποίων το θέμα αλλάζει στο $2^o$ και $3^ο$ πρόσωπο από $e \rightarrow i$. \\
 \addDVerb{Sprechen}{sprech}{spr\textcolor{red}{i}ch} \\
 \addDVerb{Nehmen}{nehm}{n\textcolor{red}{i}mm} \\
 
 \gls{Essen}, 
 
 \subsubsection{F Verben}
 Ρήματα των οποίων το θέμα αλλάζει στο $2^o$ και $3^ο$ πρόσωπο από a $\rightarrow$ ae. \\
 \addDVerb{Fahren}{fahr}{f\textcolor{red}{ae}hr} \\
 
 \gls{Waschen}, 
 
 \subsubsection{G Verben}
 Ρήματα τα οποία τελειώνουν σε -n (-eln) \\
 \addGVerb{Basteln}{bastel} \\
 \gls{Wechseln}, 
 
 \section{ ... Verben}
 \begin{tabular}{l l l l l l}
  & \gls{Sein} & \gls{Haben} & \gls{Werden} & \gls{Wissen} & \gls{Moegen} \\
  ich & bin & habe & werde & weiss & mag \\
  du & bist & hast & wirst & weisst & magst \\
  er, sie, es & ist & hat & wird & weiss & mag \\
  wir & sind & haben & werden & wissen & moegen \\
  ihr & seid & habt & werdet & wisst & moegt \\
  Sie & sind & haben & werden & wissen & moegen \\
 \end{tabular}

 
 \section{Modalverben}
 
 \begin{tabular}{l l l l l l l}
  & \gls{Moechten} & \gls{Muessen} & \gls{Koennen} & \gls{Duerfen} & \gls{Wollen} & \gls{Sollen} \\
  ich & moechte & muss & kann & darf & will & soll \\
  du & moechtest & musst & kannst & darfst & willst & sollst \\
  er, sie, es & moechte & muss & kann & darf & will & soll \\
  wir & moechten & muessen & koennen & duerfen & wollen & sollen \\
  ihr & moechtet & muesst & koennt & duerft & wollt & sollt \\
  Sie & moechten & muessen & koennen & duerfen & wollen & sollen \\
 \end{tabular}

 
 
\end{appendix}