
\begin{appendix}
 
 \section{Verben - Ρήματα}
 
 Στο παρόν παράρτημα παραθέτουμε τις συζηγίες Ρημάτων και ποια ρήματα ανήκουν σε αυτές.
 
 \subsection{Regelmaessige Verben - Ομαλά Ρήματα}
 Ρήματα που συντάσονται ομαλά. \\ 
 \addNormalVerb{\gls{Wohnen}}{wohn} \\ 
 \addNormalVerb{\gls{Kommen}}{komm} \\
 
 \gls{Abholen}, \gls{Abstellen}, \gls{Anziehen}, \gls{Anrufen},
 \gls{Anstellen},\gls{Aufhoeren}, \gls{Aufmachen}, \gls{Aufraeumen},
\gls{Aufstehen}, \gls{Aufwachen}, \gls{Ausmachen}, \gls{Aussteigen},
\gls{Ausziehen}, \gls{Backen}, \gls{Bauen}, \gls{Bedienen}, \gls{Beschreiben},
\gls{Bekommen}, \gls{Bestellen}, \gls{Bezahlen}, \gls{Bleiben}, \gls{Brauchen},
\gls{Bringen}, \gls{Buchen}, \gls{Buchstabieren}, \gls{Diskutieren},
\gls{Einkaufen}, \gls{Einpacken}, \gls{Einsteigen}, \gls{Einziehen},
\gls{Erkennen}, \gls{Erzaehlen}, \gls{Entschuldigen}, \gls{Fragen},
\gls{Fruehstuekcen}, \gls{Fotografieren}, \gls{Funktionieren}, \gls{Geben},
\gls{Gehen}, \gls{Glauben}, \gls{Gucken}, \gls{Heiraten}, \gls{Herstellen} ,
\gls{Holen}, \gls{Informieren}, \gls{Kennen}, \gls{Kochen},
\gls{Kontrollieren}, \gls{Korrigieren}, \gls{Kuendigen}, \gls{Leihen},
\gls{Lernen}, \gls{Malen}, \gls{Machen}, \gls{Meinen}, \gls{Merken},
\gls{Mitbringen}, \gls{Mitkommen}, \gls{Operieren}, \gls{Packen},
\gls{Parken}, \gls{Passieren}, \gls{Rauchen}, \gls{Reisen}, \gls{Rufen},
\gls{Sagen}, \gls{Schauen}, \gls{Schmecken}, \gls{Schreiben}, \gls{Schwimmen},
\gls{Spazieren}, \gls{Spielen}, \gls{Spuelen}, \gls{Stehen}, \gls{Stimmen},
\gls{Stoeren}, \gls{Studieren}, \gls{Suchen}, \gls{Telefonieren}, 
\gls{Trinken}, \gls{Umziehen}, \gls{Ueben}, \gls{Ueberlegen}, \gls{Verdienen},
\gls{Vergleichen}, \gls{Verstehen}, \gls{Vorbeikommen}, \gls{Wecken},
\gls{Zuhoeren}
 
 \subsection{Unregelmaessige Verben - Ανώμαλα Ρήματα}
 
 \subsubsection{B Verben}
 Ρήματα τα οποία χρειάζονται ένα -e- ανάμεσα στο θέμα (-t, -d, -m, -n, διπλό σύμφωνο) και την κατάληξη για λόγους ευφωνίας. \\ \newline
 \addBVerb{\gls{Arbeiten}}{arbeit} \\ \newline
 \gls{Antworten}, \gls{Atmen}, \gls{Baden}, \gls{Bedeuten}, \gls{Bieten}, \gls{Entscheiden}, \gls{Finden}, \gls{Kosten}, \gls{Ordnen}, \gls{Rechnen}, \gls{Schneiden}, \gls{Stattfinden}, \gls{Verbieten}, \gls{Vorbereiten}, \gls{Warten}, \gls{Zeichnen}
 
 \subsubsection{C Verben}
 Ρήματα των οποίων το θέμα λήγει σε -ss, -s, -z, -tz, -x, επομένως και χάνεται το -s- στο $2^o$ ενικό πρόσωπο. \\
 \addCVerb{\gls{Heissen}}{heiss} \\ \newline
 \gls{Boxen}, \gls{Duschen}, \gls{Essen}, \gls{Putzen}, \gls{Reisen},
\gls{Sitzen}, \gls{Tanzen} 
 
 \subsubsection{D Verben}
 Ρήματα των οποίων το θέμα αλλάζει στο $2^o$ και $3^ο$ πρόσωπο από $e \rightarrow ie$. \\
 \addDVerb{Sehen}{seh}{s\textcolor{red}{ie}h} \\
 
 \gls{Ansehen}, \gls{Aussehen}, \gls{Fernsehen}, \gls{Lesen}, 
 
 \subsubsection{E Verben}
 Ρήματα των οποίων το θέμα αλλάζει στο $2^o$ και $3^ο$ πρόσωπο από $e \rightarrow i$. \\
 \addDVerb{Sprechen}{sprech}{spr\textcolor{red}{i}ch} \\
 
 \gls{Essen}, \gls{Geben}, \gls{Helfen}, \gls{Messen}, \gls{Mitnehmen},
\gls{Treffen}, \gls{Vergessen}
 
 \subsubsection{F Verben}
 Ρήματα των οποίων το θέμα αλλάζει στο $2^o$ και $3^ο$ πρόσωπο από a $\rightarrow$ ae. \\
 \addDVerb{Fahren}{fahr}{f\textcolor{red}{ae}hr} \\
 
 \gls{Abfahren}, \gls{Anfangen}, \gls{Einladen}, \gls{Einschlafen},
\gls{Fallen}, \gls{Hinfallen}, \gls{Lassen}, \gls{Schlafen}, \gls{Waschen},
\gls{Wegfahren}
 
 \subsubsection{G Verben}
 Ρήματα τα οποία τελειώνουν σε -n (-eln) \\
 \addGVerb{Basteln}{bastel} \\
 \gls{Dauern}, \gls{Feiern}, \gls{Klingeln}, \gls{Tun}, \gls{Wechseln}, \gls{Wehtun}
 
 \section{ ... Verben}
 \begin{tabular}{l l l l l l}
  & \gls{Sein} & \gls{Haben} & \gls{Werden} & \gls{Wissen} & \gls{Moegen} \\
  ich & bin & habe & werde & weiss & mag \\
  du & bist & hast & wirst & weisst & magst \\
  er, sie, es & ist & hat & wird & weiss & mag \\
  wir & sind & haben & werden & wissen & moegen \\
  ihr & seid & habt & werdet & wisst & moegt \\
  Sie & sind & haben & werden & wissen & moegen \\
 \end{tabular}

 
 \section{Modalverben}
 
 \begin{tabular}{l l l l l l l}
  & \gls{Moechten} & \gls{Muessen} & \gls{Koennen} & \gls{Duerfen} & \gls{Wollen} & \gls{Sollen} \\
  ich & moechte & muss & kann & darf & will & soll \\
  du & moechtest & musst & kannst & darfst & willst & sollst \\
  er, sie, es & moechte & muss & kann & darf & will & soll \\
  wir & moechten & muessen & koennen & duerfen & wollen & sollen \\
  ihr & moechtet & muesst & koennt & duerft & wollt & sollt \\
  Sie & moechten & muessen & koennen & duerfen & wollen & sollen \\
 \end{tabular}
 
 
 \section{Perfect}
 O \textbf{Perfect} σχηματίζεται με τον Ενεστώτα των ρημάτων Sein (ρήματα
κίνησης ή αλλαγής κατάστασης) ή Haben (όλα τα άλλα) και τη μετοχή (Partizip
Perfekt) του ρήματος. Στα Ελληνικά μπορεί να μεταφραστεί ως 
 \begin{description}
  \item[Παρατατικός] Ο μαθητής έγραφε ...
  \item[Αόριστος] Ο μαθητής έγραψε ...
  \item[Παρακείμενος] Ο μαθητής έχει γράψει ... 
 \end{description}
 
 \begin{itemize}
  \item Ρήματα που το θέμα τους τελειώνει σε  \textbf{-t,-d,-m,-n} παίρνουν
κατάληξη \textbf{-et}.
  \item Οι παρακάτω περιπτώσεις, σχηματίζουν τη μετοχή χωρίς το \textbf{ge-}
  \begin{itemize}
    \item Ρήματα που αρχίζουν με \textbf{er-, ver-, zer-, be-, ge-}.
    \item Ρήματα που τελειώνουν σε \textbf{-ieren}.
  \end{itemize}
  \end{itemize}

 
 
 \section{Imperativ - Προστακτική}
 
 
 
\end{appendix}
